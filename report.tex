\documentclass[a4paper,11pt]{report}
\usepackage[latin1]{inputenc}
\usepackage[english]{babel}
\bibliographystyle{unsrt}

\title{Building data.eurecom.fr}
\author{Anne-Elisabeth Gazet}
\date{Fall 2011}

\begin{document}
\maketitle

\chapter*{Introduction}

\section*{Context}
The web as we know it today mainly consists of documents that are linked together. We are now moving to a new era, where raw data is being published on the web as \emph{Linked Data}, and woven into the \emph{Web of Data} or \emph{Semantic Web}. 
\begin{quotation}
``The vision of the Semantic Web is to extend principles of the Web from documents to data. Data should be accessed using the general Web architecture using, e.g., URI-s; data should be related to one another just as documents (or portions of documents) are already. This also means creation of a common framework that allows data to be shared and reused across application, enterprise, and community boundaries, to be processed automatically by tools as well as manually, including revealing possible new relationships among pieces of data.''
\begin{flushright}
from the W3C Semantic Web FAQ\footnote{http://www.w3.org/2001/sw/SW-FAQ}
\end{flushright}
\end{quotation}
In order to achieve these goals, new technologies were developed:
\begin{itemize}
\item RDF, for Resource Description Framework, is the main building block of the Semantic Web. It's a simple interchange format, where data is represented as triples of the form (subject, predicate, object) ;
\item OWL (Web Ontology Language) and RDFS (RDF Schema) are vocabularies for describing RDF properties and classes ;
\item SPARQL (SPARQL Protocol and RDF Query Language) is a protocol and query language for semantic web data sources ;
\item SPARQL Update, a companion language for SPARQL, enables modifying a data source. %(It should soon become a W3C recommendation.)
\end{itemize}

More information on the Linked Data paradigm can be found in~\cite{Heath2011}. 

\section*{Motivation for the project}
There is a trend to increase government transparency by releasing more and more public sector information on the web as linked data. The first initiative of the kind was data.gov.uk in the United Kingdom, which was followed by similar projects around the world. 

Universities and schools also generate a lot of data about students, promotions, professors, courses, publications, departments, rooms, schedules, exams, etc. The goal of this project is to take all this data generated by EURECOM, transform it in semantic formats (RDF), interlink it with other data (from other universities and datasets) and publish the whole as linked data in order to develop showcase applications that provide useful services to students and professors. %Ultimately, a mobile application could be developed.

\section*{Contents}

\chapter{Similar projects}
%Shorten this chapter, and just put short description of the projects, and the conclusions as found in the presentation from week 1 ?

This project was inspired by recent similar initiatives in other universities. I studied them in order to see what kind of data they made available, which approach they took in order to publish their data, and what kind of applications were built thanks to the newly exposed data. 
\section*{data.open.ac.uk - open linked data from The Open University}
The OU's LUCERO (Linking University Content for Education and Research Online) project\footnote{http://lucero-project.info/lb/} was the first initiative to expose public information from a university as Linked Open Data. The data.open.ac.uk platform was developed as part of the LUCERO project. 

Datasets exposed on this platform include:
\begin{description}
\item{Open Research Online:} publications from OU researchers ;
\item{OU podcasts:} collection of Audio and Video material related to education and research at the Open University ;
\item{Course Descriptions ;}
\item{OpenLearn:} metadata related to units of teaching and learning material openly available from the OpenLearn website ;
\item{KMi Planet Stories:} from the online news system of the Knowledge Media Institute ;
\item{KMi People Profiles.}
\end{description}

Vocabularies used to represent the data include:
\begin{description}
\item{BibO:} the Bibliographic Ontology\footnote{http://bibliontology.com/specification} is used to represent information about publications originating from OU researchers ;
\item{W3C Media Ontology:} this ontology\footnote{http://www.w3.org/TR/mediaont-10/} is used to describe audio and video material in the OU podcasts dataset ;
\item{AIISO and the courseware ontology:} the Academic Institution Internal Structure Ontology\footnote{http://vocab.org/aiiso/} and the courseware ontology\footnote{http://courseware.rkbexplorer.com/ontologies/courseware} are used to describe courses ;
\item{FOAF:} the Friend Of A Friend ontology\footnote{http://xmlns.com/foaf/spec/} is used to represent information on the staff members at the Knowledge Media Institute
\end{description}

Applications
\begin{itemize}
\item OpenLearn Linked Data\footnote{http://fouad.zablith.org/apps/openlearnlinkeddata/} makes use of data from data.open.ac.uk to suggest courses, podcasts and other OpenLearn units that relate to an OpenLearn Unit ;
\item The OU Expert Search\footnote{http://kmi-web15.open.ac.uk:8080/ExpertSearchClient/ (accessible inside the OU network only)} system allows users to find academics at the Open University who are experts in a given domain ;
\item OUExperts\footnote{http://vimeo.com/19743762} is a mobile (android) application to find Open Univeristy experts in a given domain, and connect to their social network ;
\item Buddy Study\footnote{http://www.matthew-rowe.com/BuddyStudy/} suggests potential contacts and Open University courses to follow for students, based on the analysis of the topics in the user's Facebook page.
\end{itemize}

\section*{data.southampton.ac.uk}
\section*{data.ox.ac.uk}
\section*{LODUM - Linked Open Data University of Münster}
\section*{National Research Council (CNR) - data.cnr.it}
\section*{Conclusion}



\chapter{Data model and design of our ontology}
Publishing data in RDF format does not make it automatically useful and reusable. It is very important to choose carefully the way the data will be modeled, and to document this. This is the most time-consuming task in such a project, but it is well worth it: a well-designed data model will help future application developers consume the data, and combine it with external datasets. 
\section{Available data}
One of the difficulties with this project is that we had to think about the data model \emph{before} actually getting access to the data, and therefore, before knowing exactly what data would be exposed. To have a clearer idea of what data was stored at EURECOM, I had a look at the internet and intranet sites of the institute. The information available there is mainly about: 
\begin{itemize}
\item people (teachers, researchers, doctoral students, staff members), 
\item research outputs, 
\item courses. 
\end{itemize}
Assuming these would be the datasets we would access later on, we designed an RDF data model. %The aim of this model is to fit the data accurately, as well as to make the data reusable by third parties. 
 
\section{Choice of external vocabularies}
To achieve re-usability, it is paramount to represent data using vocabularies that are used by other projects as well. This way, the data will be easier to consume for existing applications. Furthermore, the fact that a vocabulary was accepted by the community as a de facto standard is a good hint that this vocabulary is well designed, and well fitted to its domain. 

Yet, since there are still relatively few universities publishing their data as Linked Data, we could not rely only on popularity to make choices. Besides, each dataset being unique, there are relationships that we wanted to represent, to which we did not find any equivalent in the other projects' datasets. In this case, we used tools such as Schema-Cache\footnote{http://schemacache.com/}, Schemapedia\footnote{http://schemapedia.com/} and Sindice\footnote{http://sindice.com/} to see if there existed any term we could use. 

Alongside popularity, the other criteria we took into account to select vocabularies are:
\begin{itemize}
\item the similitude between the concepts and terms present in the vocabulary, and the concepts needed to represent the data ;
\item whether the vocabulary was created as part of a bigger project: this means it is the result of a consensus rather than the vision of a sole individual on the domain.
\end{itemize}

In the end, we selected the following vocabularies: 
\begin{itemize}
\item FOAF\footnote{http://xmlns.com/foaf/spec/}: the Friend Of A Friend vocabulary defines terms for describing persons and their relations to other people and objects. We use it to describe people at EURECOM. 
\item Dublin Core terms\footnote{http://dublincore.org/documents/dcmi-terms/}: this vocabulary is a set of generic metadata terms whose purpose is to describe numeric and physical resources. We use for example the predicate dc:creator to represent the relationship between a document and its creator. 
\item Participation\footnote{http://vocab.org/participation/schema}: The participation ontology is a simple model for describing the roles that people play within groups. As discussed later, we subclass the class part:Role in the REVE ontology, to describe roles played by people inside EURECOM. 
\item AIISO\footnote{http://vocab.org/aiiso/schema}: The Academic Institution Internal Structure Ontology provides classes and properties to describe the internal organizational structure of an academic institution. We use the aiiso:Course and aiiso:KnowledgeGrouping classes to define courses and tracks. 
\item BIBO\footnote{http://bibliontology.com/specification}:  The Bibliographic Ontology describe bibliographic things on the semantic Web in RDF. It has been inspired by many existing document description metadata formats. We use it to describe the articles in EURECOM's scientific publications repository. 
\item LODE\footnote{http://linkedevents.org/ontology/}: The ontology for Linking Open Descriptions of Events is an ontology for publishing descriptions of historical events as Linked Data, and is designed to be compatible with other event-related vocabularies and ontologies. We use the predicates lode:atTime, lode:atPlace and lode:involvedAgent to describe course sessions. 
%\item DUL
\item OWL-Time\footnote{http://www.w3.org/TR/owl-time/}: This ontology provides a vocabulary for expressing facts about topological relations among instants and intervals, together with information about durations, and about datetime information. We use it to describe the temporal aspects of course sessions. 
\item Rooms\footnote{http://vocab.deri.ie/rooms}: It's simple vocabulary for describing the rooms in a building, which we use to describe the rooms and buildings where courses take place. 
\end{itemize}


\section{REVE - the Research and Education Vocabulary for EURECOM}
Some of the concepts we need to represent the data are specific to EURECOM, so we defined a set of new terms. When it was possible, we defined these terms as extensions of existing terms. 

The vocabulary is specified using the RDFS and OWL languages. The latest specification can be found in RDF format at the following address: \texttt{https://github.com/aegazet/REVE-Ontology}

\subsection{Modeling issue: what is a course ?} %TODO:title
One of the main modeling issues we encountered was due to a particularity of the french language: the word ``cours'' can mean many different things, even when we restrict ourselves to the context of a school. For instance, it can designate a lecture, or a course. We also had to make the distinction between a course which took place in a given semester, and the general subject of the course. E.g., ``WebSem - Spring 2010'' and ``WebSem - Spring 2011'' are two modules which both treated the same subject --- an introduction to the Semantic Web --- but they took place in different times, and different people were registered for them. 

In such cases, it is essential to discuss the issue with experts on the domain we try to model. Here we asked Alexia Cepero, the Student's Pedagogy Officer of EURECOM, what was the correct representation. This led us to define the two following classes:
\begin{itemize}
\item Course: \emph{a teaching unit. A course is composed of several course sessions.}
\item Course session: \emph{a course session is a punctual event on which teacher and students gather, for a given course.}
\end{itemize}
Counter intuitive as it might seem, there is no concept representing the unity of subject between, for instance, ``WebSem - Spring 2010'' and ``WebSem - Spring 2011''. But in the data we publish, we can still hint at a relationship between the two instances, for example by stating they have the same \texttt{aiiso:code} attribute. 

\subsection{Representing an n-ary relation}
Another issue we encountered, but was easier to solve, resulted from an assumption I made on the data which turned out to be false. I had assumed the credits one can earn by successfully completing a course only depended on the course itself. It turns out that it also depends on the track of study the student is following at EURECOM. 

In other words, what I had modeled as a (course, credit) relationship is actually a (course, credit, track) relationship. The minor trouble with this, is that such a relationship cannot be represented with a single triple. 

This is a common situation in ontology modeling. Good modeling practices have therefore been identified by the community: ontology patterns for representing n-ary relations in RDF and OWL are presented in a W3C working group note\footnote{http://www.w3.org/TR/swbp-n-aryRelations/}. We picked the first pattern described in this note, which consists in creating a new class and n new properties to represent the n-ary relation. 
%insert graph representing the result

\section{First data model} \label{sec:1stdatamodel}
This first model was developed having in mind the data I had observed on the various sites of the school. Therefore, it is much more complete than what has been implemented so far, mainly because a lot of data has not been exposed yet. This preliminary model is documented here in the hope that it will be useful for future developments of the project. 

You will find as annexes several graphs illustrating this model. Each graph is focused on a particular domain: persons, documents, and courses. 

\subsection{Persons}
This model was inspired by data that is visible on EURECOM's public staff directory. 

\subsection{Documents}
This model is mainly a subset of the Bibliographic Ontology. 
%transition !?
\subsubsection*{Representing a list of authors}
There are two options for representing an article's list of authors through the predicate \texttt{bibo:authorList}: the range of this property is the union of the classes  \texttt{rdf:Seq} and  \texttt{rdf:List}. Both of these classes are \emph{container} classes, that is to say they were designed to represent sets of objects, which is not an easy task with triples. It is generally advised not to use container classes in RDF, unless the order of the elements is important. Here the order does matter, so we had a closer look at those classes to make our choice: 
\begin{itemize}
\item \texttt{rdf:List} is a linked list: instances of this class have an \texttt{rdf:first} attribute pointing at the \emph{head} of the list, and an \texttt{rdf:rest} attribute pointing at the \emph{tail} of the list (another \texttt{rdf:List}). The end of the list is indicated by setting the \texttt{rdf:rest} to \texttt{rdf:nil}. 
\item \texttt{rdf:Seq} is more like a table: an instance of this class has attributes of the form \texttt{rdf:{\_}1}, \texttt{rdf:{\_}2}, ... \texttt{rdf:{\_}n} pointing at the elements of the list. 
\end{itemize}

The two approaches have upsides and downsides. Using \texttt{rdf:List}, you can indicate exactly what the elements of the list are, which you cannot with \texttt{rdf:Seq}. Indeed the Semantic Web is built on an \emph{open world assumption}: the absence of a statement does not mean that the statement is false. In other words, you can state that an \texttt{rdf:Seq} has \emph{at least} n elements, but you cannot state that there are no other elements. 

On the other hand, if you use \texttt{rdf:List} to represent your data, you will need a lot of instances of \texttt{rdf:List}. This makes the structure of the data complicated, and in general hard to query. The more so as, since it makes no sense to give a globally unique identifier to every \texttt{rdf:List} in your data, this approach leads to creating \emph{blank nodes} for them. Blank nodes are nodes without a URI, and it is good practice to avoid them whenever possible, as stated in~\cite{Heath2011}:
\begin{quotation}
``The scope of blank nodes is limited to the document in which they appear, meaning it is not possible to create RDF links to them from external documents, reducing the potential for interlinking between different Linked Data sources. In addition, it becomes much more difficult to merge data from different sources when blank nodes are used, as there is no URI to serve as a common key.''
\end{quotation}

This motivated our final decision to use \texttt{rdf:Seq} to represent the list of authors. 
% -> for future developments of the project
%graphs, graphs, graphs
\section{Present state of the data}
% -> to start consuming data now
%graph of what can actually be found in the triplestore at the moment
As stated in \ref{sec:1stdatamodel}, the data that has been translated in RDF so far in this project only fits a subset of the general data model we imagined. The graph 
%insert ref to graph
in annex represents this subset. To start consume data now, application developers can refer to this graph to know what triples they can find in the triplestore. 

This kind of documentation is essential to make data reusable. Sadly, as I looked around for similar initiatives, I noticed it is often missing, even in larger scale projects. Let us remind here that publishing data in RDF format does not make it automatically reusable: if there is no information regarding the data model and the vocabulary used, the application developer has no way of guessing what information is available, and how to retrieve it. 



\chapter{Data conversion}
Once the data model has been defined, we can move on to a more concrete stage: the translation of data into RDF. 
%A word on naming the entities somewhere
%Quick word on R2RDF somewhere
\section{Getting the data} 
%Quickly describe the web service, show example of JSON
In the early stages of the project, the plan was to give me access to a dump of a relational database containing the interesting information. Yet for various reasons (practical, legal, ...) it was decided after a few weeks that the IT department would instead set up a web service serving JSON files for me to process. 

In the weeks preceding this change of strategy, I had started to get familiar with R2RML, a language for expressing mappings from relational databases to RDF datasets\footnote{http://www.w3.org/TR/r2rml/}. I tested in particular db2triples\footnote{http://blog.antidot.net/2011/10/07/db2triples-une-implementation-de-r2rml-et-directmapping-en-open-source/}, an open-source implementation of R2RML by the french firm Antidot. In the end, this was not used for the project, but could be very useful for future developments. 

The JSON files served by the web service have the following format: 
\begin{verbatim}
{"status":"successful",
"request_uri":"http:\/\/intranet-v2.eurecom.fr\/data\/...",
"label":"Courses list - 2004",
"count":0,
"response":[]}
\end{verbatim}
The \texttt{status} attribute states whether or not the file could be retrieved successfully. The data is in the \texttt{response} attribute. The format of the response depends on the file. The different types of files presently available are: 
\begin{itemize}
\item course: files describing courses contain the full and abbreviated names of the courses, the list of teachers for a given course, its abstract, ...
\item course session: files describing course sessions repeat some of the information given in course files, and contain info on the sessions (date and hour, room and teacher involved)
\item publications:  these files contain articles' title, list of authors, the name of the conference where an article was presented, as well as the date and place of the conference
\item teacher, researcher, doctor, phd: these files describe people in terms of the \emph{role(s)} they hold or have held at EURECOM
\end{itemize}

%So I present it here shortly, before presenting the web service. 
%\subsection{R2RML: RDB to RDF Mapping Language}

\section{Building RDF triples with rdflib}
%Justify choice of Python : easy and fast to test things, good existing tools to work with RDF, and for now we don't care much about fast-executing code since the conversion is done once and for all - it's more about testing feasibility
\subsection{Why Python?}
As Python is a very popular scripting language, there exist good tools to work with RDF in Python. Besides, with Python it is easy and fast to test things, and easy to work with JSON files. The drawback of using a scripting language rather than, say, Java, or any other compiled language, is a lesser speed of execution. But at this stage of the project, the conversion is done once and for all -- it is more about testing feasibility -- so I did not worry much about execution speed, and chose to use Python and the rdflib library to make the data conversion.

\subsection{The rdflib library}
%A few words on rdflib
The rdflib package is intended to provide core RDF types and interfaces for working with RDF. 

The primary interface that rdflib exposes for working with RDF is \texttt{Graph}. rdflib graphs have ordinary set operations (e.g. \texttt{add()} to add a triple) plus methods that search triples and return them in arbitrary order. They are best thought of as a set of 3-item triples. 

Elements of a triple are created using the following classes:
\begin{itemize}
\item \texttt{URIRef} for URI references
\item \texttt{Literal} for literals (datatypes and language tags are supported)
\item \texttt{BNode} for blank nodes
\end{itemize}

With extensions provided in the rdfextras library, it is possible to store the triples directly in a triplestore. Here, I used the serialization capacities of rdflib: a graph is easily serialized in RDF/XML format, and saved in a file.

 Moreover, rdflib provides a \texttt{parse()} method which not only loads a graph from a file, but can also retrieve files from the web. 

rdflib also has a convenient \texttt{Namespace} class, with which fully qualified URIs in the namespace can be constructed by accessing a \texttt{Namespace} object like a dictionary:
\begin{verbatim}
>>> DCT = rdflib.Namespace("http://purl.org/dc/terms/")
>>> DCT["creator"]
rdflib.term.URIRef('http://purl.org/dc/terms/creator')
\end{verbatim}

More information on how to use rdflib is available on the library's documentation pages\footnote{http://readthedocs.org/docs/rdflib/en/latest/}.

\subsection{URI scheme}
I took the following conventions to assign URIs to the entities described in the data:
\begin{itemize}
\item the base URL for the dataset is \texttt{http://data.eurecom.fr/}
\item every resource gets a URI of the form: \\\texttt{http://data.eurecom.fr/<domain>/<id>}  \\where \texttt{<domain>} corresponds to the type of the entity, and is one of the following:
\begin{itemize}
\item \texttt{course}
\item \texttt{semester}
\item \texttt{people}
\item \texttt{publication}
\item \texttt{session}
\item \texttt{room}
\item \texttt{track}
\item \texttt{role}
\item \texttt{department}
\item \texttt{event}
\end{itemize}
\item for the  \texttt{<id>} part of the URI, I generally used the integers that identify the entities in the original database records, and are present in the JSON files. 
\item for the conferences, there are no such identifiers in the database, only strings of characters describing the event (e.g.``ISM 2011, IEEE International Symposium on Multimedia, December 5-7, 2011, Dana Point, CA, US''), so the identifier is the MD5 hash of this string.
\end{itemize}

%Snippets of code, explain conversion strategy
\subsection{Conversion strategy}
The conversion process is pretty simple. It boils down to reading sequentially the input JSON files, and apply the ``rules'' defined in our RDF data model, and in the URI scheme. 

First thing to do is to get a file from the web service, using the \texttt{urllib} module. The file is then parsed into a Python dictionary using the \texttt{json} module, and the elements of the \texttt{response} array are parsed individually. 

In general, an RDF triple is generated for every attribute of every element, then added to one of the following graphs: courses, sessions, semesters, tracks, people, roles, publications, events, rooms, departments. This splitting of the triples in several graphs results in the creation of several RDF files, each containing information on a particular domain. 

\section{Linking data to an external source: GeoNames}
%Explain choice of source : widely used service, URIs for the resources that can be built using their geocoding webservice, thus fitting the data we had at hand (country code and city name)
At this point of the data conversion, we have information about articles that were presented at conferences all around the world. But we do not have much geographical information about the conferences. The JSON files contain, for each article, the name of the city where the conference took place, as well as the name and the ISO two-letter code for the country:
\begin{verbatim}
city": "Cape Town",
"country": {
    "id": "country/za",
    "shortlabel_en": "SOUTH AFRICA",
    "longlabel_en": null
}
\end{verbatim}

This information is enough to do some geocoding\footnote{from Wikipedia: \emph{Geocoding is the process of finding associated geographic coordinates (often expressed as latitude and longitude) from other geographic data, such as street addresses, or zip codes (postal codes). }}, so as, for example, to be able to represent the conferences' locations on a map. The geocoding service I used is GeoNames\footnote{http://www.geonames.org/}, a popular geographical database available and accessible through various Web services. Each geographical entity it describes has a unique integer assigned to it, as well as a URI to be used in semantic web applications. GeoNames data are linked to DBPedia, the project publishing content from Wikipedia as linked data. 

%Show code and GeoNames's WS URL
The geocoding service is available through a REST API. This is the query to retrieve information about Cape Town, South Africa:\\
\texttt{http://ws.geonames.org/searchJSON?q=\%22Cape\%20Town\%22\&country=za}

And this is the response, in JSON format:
\begin{verbatim}
{
  "totalResultsCount": 43,
  "geonames": [
    {
      "countryName": "South Africa",
      "countryCode": "ZA",
      "lng": 18.423218,
      "name": "Cape Town",
      "geonameId": 3369157,
      "lat": -33.925839,
      ...
    }
  ]
}
\end{verbatim}

From this response, I extract the geonameID, use it to construct the URI for the place, using the scheme: \texttt{http://sws.geonames.org/<geonameID>}, and use this URI reference in new triples. I also store the latitude and longitude in RDF format for later use. Indeed, GeoNames does not provide an official SPARQL endpoint, so every application developer willing to get their data directly in RDF format has to replicate the data locally (or to use an unofficial SPARQL endpoint.)
%They don't have their own official SPARQL endpoint so we store their (lat, long) data locally. (Would it be worth it to store everything ?)
\section{Loading triples in a triplestore using SPARQL Update}

The last step of the conversion consists in automating the loading of the triples into a triplestore.

To do so, I used the SPARQL Update language. It is not a W3C recommendation yet, but should soon become one. It has been mature for some time now. Consequently, most of the popular triplestores support it. 

The principle is simple: the update request is sent to the SPARQL endpoint via the HTTP POST method by URL encoding the parameters. Here I use the LOAD command to load a local file into the triplestore:
\begin{verbatim}
LOAD <file:// ... > INTO GRAPH <http://data.eurecom.fr/rooms/>
\end{verbatim}

The triplestore is now loaded with the newly generated triples, and ready to answer queries. 
%At the time of writing, I have only tested my script on a local Sesame server

\section{How to use the scripts}
%TODO: implement cycling over the geonames servers, merge the 2 scripts, then merge with the loading ?
%TODO: accept parameters for the script, at least to specify the destination sparql endpoint


\chapter{Showcase application : a map of EURECOM publications}
%Intro: there are no killer app for the SW, it enables to build many new services, it's up to the programmer to get inspiration from the data
%There is no \emph{killer app} for the Semantic Web. Rather, it helps build richer applications than before, and much faster than before. 
To demonstrate what can be done with the newly generated triples, I developed a map of the world with pins showing every conference where EURECOM researchers presented a paper. This was made easy thanks to the SIMILE Exhibit project. Exhibit\footnote{http://simile-widgets.org/exhibit/} is a tool to create data visualization web pages. With a bit of experience, one can create an Exhibit page within an hour or two. 

The capital thing that is required to use Exhibit, is to prepare your data: it has to be in a particular JSON format. The minimal Exhibit JSON file contains an array named \texttt{items}, whose elements have at least a \texttt{label} and a \texttt{type} attribute:
\begin{verbatim}
{  items: [
    {  label:  "Jane Smith",
       type:   "Person"  }
  ]
}
\end{verbatim}

Some data importing services are proposed, through the SIMILE Babel project\footnote{http://simile.mit.edu/babel/}. Unfortunately, the conversion from RDF/XML (or another serialization like N3) is poorly made. In particular, it fails to use the URI of an RDF term as the identifier for that resource in the JSON file. As a consequence, all the \emph{links} in the linked data get broken. So I resorted to implementing the conversion myself, in Python again, with help from the \texttt{SPARQLWrapper} module. 

First, all data about papers that were presented at conferences are retrieved by issuing the following SPARQL query:
\begin{verbatim}
PREFIX rdf:<http://www.w3.org/1999/02/22-rdf-syntax-ns#>
PREFIX rdfs:<http://www.w3.org/2000/01/rdf-schema#>
PREFIX dct:<http://purl.org/dc/terms/>
PREFIX foaf:<http://xmlns.com/foaf/0.1/>
PREFIX lode:<http://linkedevents.org/ontology/>
PREFIX geo:<http://www.w3.org/2003/01/geo/wgs84_pos#>
PREFIX bibo:<http://purl.org/ontology/bibo/>

SELECT DISTINCT 
  ?pub ?title ?date ?conf 
  ?conf_title ?lat ?long ?author
WHERE {
  ?pub rdf:type foaf:Document .
  ?pub dct:title ?title .
  ?pub dct:date ?date .
  ?pub bibo:presentedAt ?conf .
  ?conf rdfs:label ?conf_title .
  ?conf lode:atPlace ?place .
  ?place geo:lat ?lat .
  ?place geo:long ?long .
  ?pub dct:creator ?author
}
\end{verbatim}
The response, in JSON format, is then transformed and put into a Python dictionary. A second query is issued to get information about the authors: 
\begin{verbatim}
PREFIX rdf:<http://www.w3.org/1999/02/22-rdf-syntax-ns#>
PREFIX rdfs:<http://www.w3.org/2000/01/rdf-schema#>
PREFIX dct:<http://purl.org/dc/terms/>
PREFIX foaf:<http://xmlns.com/foaf/0.1/>
PREFIX part:<http://purl.org/vocab/participation/schema#>

SELECT DISTINCT 
  ?author ?fn ?ln ?roletype ?depname
WHERE {
  ?pub rdf:type foaf:Document .
  ?pub dct:creator ?author .
  ?author foaf:firstName ?fn .
  ?author foaf:lastName ?ln .
  ?author part:holder_of ?role . 
  ?role rdf:type ?roletype . 
  OPTIONAL {
    ?role part:role_at ?dep . 
    ?dep rdfs:label ?depname
  }
}\end{verbatim}
The response is also parsed in the dictionary. The dictionary gets enriched by some information specific for Exhibit to work with the data, then is serialized into a JSON file. 

The resulting application can currently be viewed at the following address:
\texttt{perso.telecom-paristech.fr/\textasciitilde{}gazet/map.html}
%Screenshots
%Needed improvements



\chapter{Future work}
%Improve on architecture
%Expose more data
%Clean up data
%Align with other datasets
%Build more applications
\bibliography{biblio}
\end{document}
